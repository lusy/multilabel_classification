\documentclass{scrartcl}
\usepackage[utf8]{inputenc}
\usepackage[T1]{fontenc}
\usepackage[ngerman]{babel}
\usepackage{amsmath}
\usepackage{lmodern} % bessere Schrift für PDFs
\usepackage{graphicx}
\usepackage{listings}
\usepackage{color}
\usepackage{comment}
\usepackage{float}

%\usepackage{lipsum} % Nur zum Auffüllen von Text

% add subdirectory for images
\graphicspath{{figures/}}


% Hurenkinder und Schusterjungen verbieten
% wtf???????
\clubpenalty=10000
\widowpenalty=10000

% neue Absatz wird mit Leerzeile begonnen:
\parindent 0pt
\parskip 12pt

\title{Multilabel Klassifikation mit Support Vector Machines und Latent Dirichlet Allocation als Featurereduktion}
\author{AutorInnen}

\date{\today}


\begin{document}

\maketitle

\begin{abstract}
    %Multilabel-Textklassifikation mit Support Vector Machines und Latent Dirichlet Allocation zur Dimensionsreduktion.
    Diese Arbeit beschreibt die Ergebnisse einer Multilabel-Textklassifikation von mathematischen Publikationen aufgrund deren Titeln und Abstracts.
    Es wurden zwei verschiedene Ansätze zur Multilabel-Klassifikation umgesetzt und miteinander verglichen.
    Die Publikationen wurden einmal direkt als Vektor-Space-Modell dargestellt und mit Hilfe von Support Vector Machines (SVM) klassifiziert.
    Der zweite Ansatz implementiert als Zwischenschritt eine Dimensionsreduktion mit Latent Dirichlet Allocation (LDA).
    Die durchgeführten Experimente zeigen, dass der direkte Ansatz mit SVM bessere Ergebnisse liefert als das Verfahren, das LDA benutzt.
\end{abstract}

\tableofcontents
\newpage

\section{Einleitung}

\begin{comment}
Einführung und Einbettung des Problems
Was ist die konkrete Problemstellung / Fragestellung der Arbeit
Literature Review
\end{comment}

Die Klassifikation von Texten ist in den letzten Jahren mit der steigenden Anzahl an Informationen und Entstehung von Textsammlungen zur einem wichtigen Verfahren zur Organisation und Handhabung geworden.
Viele praktische Anwendungen nutzen Textklassifikation als Verfahren, um Texte für Benutzer besser nutzbar zu machen, so zum Beispiel Bibliothekssysteme, Empfehlungssysteme, Spamerkennung, Erkennung der Sprache in Texten oder der Erkennung von Themen in Nachrichten.

Traditionell wurde jedem Dokument exakt ein Label $l$ aus einer Menge von Labels $L$ zugewiesen.
Wenn $|L| = 2$, dann spricht man von einem binären Klassifiaktionsproblem und wenn $|L| > 2$ von einem Multiclass Problem.
Bei der Multilabeltextklassifikation werden Dokumenten eine Menge von Labels $Y \in L$ zugeordnet.
Die Multilabelklassifikation hat auch ihren Ursprung in der Textklassifikation, da dort häufig die Anforderung besteht Dokumenten in verschiedene Gruppen einzuteilen.

In dieser Arbeit werden wir die Auswirkungen von der Latent Dirichlet Allocation zur Vorverarbeitung für die Multilabelklassifikation mit Support Vector Machines betrachten.



\input{02-methods}
\input{03-results}
\input{04-discussion}
\input{05-conclusion}

\nocite{*}
\bibliography{literature}
\bibliographystyle{alpha}

\end{document}
