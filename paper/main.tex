\documentclass{scrartcl}
\usepackage[utf8]{inputenc}
\usepackage[T1]{fontenc}
\usepackage[ngerman]{babel}
\usepackage{amsmath}
\usepackage{lmodern} % bessere Schrift für PDFs
\usepackage{graphicx}
\usepackage{listings}
\usepackage{color}
\usepackage{comment}
\usepackage{float}

%\usepackage{lipsum} % Nur zum Auffüllen von Text

% add subdirectory for images
\graphicspath{{figures/}}


% Hurenkinder und Schusterjungen verbieten
% wtf???????
\clubpenalty=10000
\widowpenalty=10000

% neue Absatz wird mit Leerzeile begonnen:
\parindent 0pt
\parskip 12pt

\title{Multilabel Klassifikation}
\author{AutorInnen}

\date{\today}


\begin{document}

\maketitle

\begin{abstract}
    %Multilabel-Textklassifikation mit Support Vector Machines und Latent Dirichlet Allocation zur Dimensionsreduktion.
    Diese Arbeit beschreibt die Ergebnisse einer Multilabel-Textklassifikation von mathematischen Publikationen aufgrund deren Titeln und Abstracts.
    Es wurden zwei verschiedene Ansätze zur Multilabel-Klassifikation umgesetzt und miteinander verglichen.
    Die Publikationen wurden einmal direkt, ohne Vorverarbeitung, mit Hilfe von Support Vector Machines klassifiziert.
    Der zweite Ansatz implementiert als Zwischenschritt eine Dimensionsreduktion mit Latent Dirichlet Allocation.
    Ergebnisse: ..??
\end{abstract}

\tableofcontents
\newpage

\section{Einleitung}

\begin{comment}
Einführung und Einbettung des Problems
Was ist die konkrete Problemstellung / Fragestellung der Arbeit
Literature Review
\end{comment}

Die Klassifikation von Texten ist in den letzten Jahren mit der steigenden Anzahl an Informationen und Entstehung von Textsammlungen zur einem wichtigen Verfahren zur Organisation und Handhabung geworden.
Viele praktische Anwendungen nutzen Textklassifikation als Verfahren, um Texte für Benutzer besser nutzbar zu machen, so zum Beispiel Bibliothekssysteme, Empfehlungssysteme, Spamerkennung, Erkennung der Sprache in Texten oder der Erkennung von Themen in Nachrichten.

Traditionell wurde jedem Dokument exakt ein Label $l$ aus einer Menge von Labels $L$ zugewiesen.
Wenn $|L| = 2$, dann spricht man von einem binären Klassifiaktionsproblem und wenn $|L| > 2$ von einem Multiclass Problem.
Bei der Multilabeltextklassifikation werden Dokumenten eine Menge von Labels $Y \in L$ zugeordnet.
Die Multilabelklassifikation hat auch ihren Ursprung in der Textklassifikation, da dort häufig die Anforderung besteht Dokumenten in verschiedene Gruppen einzuteilen.

In dieser Arbeit werden wir die Auswirkungen von der Latent Dirichlet Allocation zur Vorverarbeitung für die Multilabelklassifikation mit Support Vector Machines betrachten.



\section{Grundlagen}
\subsection{Multilabel Klassifikation}
\label{sub:multilabel_klassifikation}
Es existieren zwei Arten von Ansätzen, um das Multilabel Klassifikation Problem zu lösen.
Bei der ersten wird das Problem transformiert und bei der zweiten werden existierende Verfahren adaptiert.
% comment lusy: verstehe ich nicht

\subsection{Support Vector Machines}
\label{sub:support_vector_machines}
Die \emph{Support Vector Machine} (SVM) ist ein Verfahren des überwachten Lernens, das zum Klassifizieren von Objekten verwendet wird.
Zum Lernen einer SVM werden die Trainingsdaten als Vektoren, zusammen mit den zugehörigen Klassen übergeben.
Für Texte ist die Repräsentation als Vektor meistens das Vektorspace-Modell, bei dem jedes Wort in dem Text als Feature angesehen wird.
Die SVM versucht dann anhand dieser Daten eine Hyperebene so in den Raum zu legen, so dass die Trainingsdaten in zwei Klassen eingeteilt werden.
Der Abstand der Vektoren, die der Hyperebene am nächsten liegen, wird dabei maximiert.
Ursprünglich wurden SVMs zur Unterteilung der Objekte in nur zwei Klassen konzipiert.
Es wurde gezeigt, dass Support Vector Machines eines der besten Verfahren zur Klassifikation sind \cite{Joachims:1998:TCS:645326.649721}.


%\begin{figure}[h]
    %\centering
    %\def\svgwitdth{0.75\columnwidth}
    %\input{figures/svm_intro.pdf_tex}
    %\caption{Hypereben einer SVM im 2-dimensionalen}
    %\label{fig:svm_intro}
    %\footnote*{http://commons.wikimedia.org/wiki/File:Svm\_intro.svg}
%\end{figure}



Um Support Vector Machines für das Multilabel Klassifizierungsproblem zu adaptieren, wird für jede Klasse $c \in L$ ein $\tilde c$ bestimmt, sodass für ein Dokument $D_i$ mit den Labels $L_i$ gilt:
\[
    \tilde c =
    \begin{cases}
        1 &\mbox{wenn } c \in L_i \\
        0 &\mbox{sonst}
    \end{cases}
\]

wobei $\tilde c$ die neue zu lernende Klasse für den binären Klassifkator repräsentiert.
Zur Klassifikation von neuen ungesehen Dokumenten wird nun jeder der binären Klassifikatoren gefragt, ob das aktuelle Dokument zu der gelernten Klasse gehört.
Ist die Antwort positiv, wird sich für dieses Dokument diese Klasse notiert, so dass man am Ende eine Menge von Klassen für das Dokument erhält.


\subsection{Latent Dirichlet Allocation}
\label{sub:latent_dirichlet_allocation}

Die \emph{Latent Dirichlet Allocation} (LDA) ist ein generatives stochastisches Modell, das eine Topic-Verteilung für eine angegebene Menge von Textdokumenten errechnet.
Dabei ist jedes Topic eine Verteilung über allen in der Dokumentensammlung vorhandenen Wörtern.
Jedes Dokument kann als eine Verteilung über Topics dargestellt werden.

% jedes Topic ist eine Verteilung über Wörter
% jedes Dokument ist Verteilung über Topics
% Latent Dirichlet Allocation(LDA)
%      generativer Prozess zum finden von Topics der Dokumente
%      findet k Topics auf Dokumenten



\subsection{Evaluierungsmaße}
Um die erzielten Ergebnisse auswerten und mit einander vergleichen zu können, haben wir die gängigen Maße \emph{Precision}, \emph{Recall} und \emph{F1} genutzt.
Vollständigkeitshalber werden hier nochmal die entsprechenden Gleichungen aufgeführt.

Precision \[\frac{\#(\text{relevant items retrieved})}{\#(\text{retrieved items})}\]
Recall \[\frac{\#(\text{relevant items retrieved})}{\#(\text{relevant items})}\]
F1-Measure \[F_1 = 2 \cdot \frac{\mathrm{precision} \cdot \mathrm{recall}}{\mathrm{precision} + \mathrm{recall}}\]

Ferner wurden zwei verschiedene Ansätze genutzt, um die Ergebnisse für ein Verfahren über alle Klassen zusammenzufassen.
Es wurden über alle Klassen sowohl die Micro- als auch die Macro-Precision, Recall und F1-Measure berechnet.
Die Macro-Average-Maße berechnen einen einfachen Durchschnitt über alle Klassen, während die Micro-Average-Maße die Klassen nach deren Größen gewichten \cite{Manning:2008:IIR:1394399}.

% TODO: micro/macro average F1-Measure --> nachdem P. die Ergebnisse korrigiert hat
%"Macroaveraging computes a simple average over classes.....
%The differences between the two methods can be large. Macroaveraging
%gives equal weight to each class, whereas microaveraging gives equal weight
%to each per-document classification decision."

\section{Methoden}

\subsection{Datensatz}
Der Datensatz besteht aus $1.1$ Millionen mathematischen Publikationen mit Title, Abstract, Klassen und Erscheinungsjahr. Die Publikationen werden in 14 verschiedene Klassen eingeteilt. Für $75.6 \%$ der Dokumente wurde nur eine Klasse vergeben, für $24.3 \%$ zwei Klassen und $0.1 \%$ mehr als zwei Klassen. Jedes Dokument wird mindestens einer Klasse zugeordnet.
\label{sub:datensatz}
\begin{table}[h]
    \centering
    \begin{tabular}{l|l|l}
        \textbf{Klasse} & \textbf{Bedeutung} & \textbf{Häufigkeit}\\
        \hline
        AC & Automation \& Control Systems & 67450\\
        EV & Computer Science, Interdisciplinary Applications & 122025\\
        EW & Computer Science, Software Engineering & 70796\\
        EX & Computer Science, Theory \& Methods & 136456\\
        MC & Mathematical \& Computational Biology & 41590\\
        PE & Operations Research \& Management Science & 85476\\
        PN & Mathematics, Applied & 266198\\
        PO & Mathematics, Interdisciplinary Applications & 82404\\
        PQ & Mathematics & 255421\\
        PS & Social Sciences, Mathematical Methods & 25679\\
        QL & LOGIC & 1448\\
        UR & Physics, Mathematical & 149917\\
        VS & Psychology, Mathematical & 8439\\
        XY & Statistics \& Probability & 104227\\
    \end{tabular}
    \caption{Bedeutung und Häufigkeit der Klassen im Datensatz}
\end{table}



\subsection{Test- und Trainingsdaten}
\label{sub:test_und_trainingsdaten}
Zunächst haben wir unseren Datensatz in Test- und Trainingsdaten im Verhältnis 3:7 aufgeteilt, um unser gelerntes Modell auf ungesehen Daten zu testen. für die Aufteilung wurde das Stratified Sampling verwendet, sodass die Klassen anteilig in Test- und Trainingsdaten vertreten sind. Wir verwenden anstatt der einzelnen Klassen, die vergebenen Klassenkombinationen, sodass die Kombinationen von Klassen anteilig in den beiden Datensätzen auftreten.

% subsection test_und_trainingsdaten (end)

\section{Ergebnisse}

\begin{figure}
    \centering
    \def\svgwitdth{0.1\columnwidth}
    \input{figures/eval_svm_labelwise_plot.pdf_tex}
    \caption{Ergebnisse der Klassifizierung mit Topics als Vorverarbeitung}
    \label{fig:svm_topic_eval}
\end{figure}
\begin{figure}
    \centering
    \def\svgwitdth{0.1\columnwidth}
    \input{figures/text_svm.pdf_tex}
    \caption{Ergebnisse der Klassifizierung Mit Unigram SVM}
    \label{fig:svm_text_eval}
\end{figure}

\begin{table}[h]
    \begin{tabular}{rcc}
        \tiny \textbf{Metrik} & \tiny\textbf{Unigram SVM Labelkombinationen} & \tiny\textbf{Unigram SVM OVR}\\
        \hline
        \tiny \textbf{Mean Precision Score}  & \tiny 58.1 & \tiny  58.4\\
        \tiny \textbf{Mean Recall Score}     & \tiny 54.9 & \tiny 43.3\\
        \tiny \textbf{Macro F1 Score}        & \tiny 56.5 & \tiny 49.7\\
        \tiny \textbf{Micro Precision Score} & \tiny 62.2 & \tiny 56.2\\
        \tiny \textbf{Micro Recall Score}    & \tiny 61.5 & \tiny 53.8\\
        \tiny \textbf{Micro F1 Score}        & \tiny 61.8 & \tiny 55.0\\
    \end{tabular}
    \caption{Ergebnisse der SVM mit Topics und Labelkombinationen}
    \label{tab:unigram_svm}
\end{table}
\begin{table}[h]
    \begin{tabular}{rccccccc}
        \tiny\textbf{Metrik} & \tiny\textbf{10 Topics} &\tiny \textbf{40 Topics} &\tiny \textbf{70 Topics} &\tiny \textbf{100 Topics} & \tiny \textbf{130 Topics} &  \tiny \textbf{160 Topics} &  \tiny \textbf{190 Topics} \\
        \hline
        \tiny \textbf{Mean Precision Score}  & \tiny 9.19 & \tiny 8.90& \tiny \textbf{9.88}&\tiny 8.47&\tiny 9.58&\tiny 9.20&\tiny 9.19\\
        \tiny \textbf{Mean Recall Score}     & \tiny 8.20 & \tiny 8.35& \tiny \textbf{8.44}&\tiny 7.89&\tiny 8.30&\tiny 8.05&\tiny 7.96\\
        \tiny \textbf{Macro F1 Score}        & \tiny 8.67 & \tiny 8.62& \tiny \textbf{9.11}&\tiny 8.17&\tiny 8.89&\tiny 8.59&\tiny 8.53\\
        \tiny \textbf{Micro Precision Score} & \tiny 11.3 & \tiny 11.3& \tiny 11.5&\tiny \textbf{18.8}&\tiny 11.4&\tiny 10.7&\tiny 10.9\\
        \tiny \textbf{Micro Recall Score}    & \tiny 8.88 & \tiny 8.66& \tiny 9.07&\tiny \textbf{14.8}&\tiny 8.98&\tiny 8.47&\tiny 8.58\\
        \tiny \textbf{Micro F1 Score}        & \tiny 9.95 & \tiny 9.70& \tiny 10.1&\tiny \textbf{16.6}&\tiny 10.0&\tiny 9.49&\tiny 9.61\\
    \end{tabular}
    \caption{Ergebnisse der SVM mit Topics und Labelkombinationen}
    \label{tab:topics_svm_labelcombs}
\end{table}

\begin{table}[h]
    \begin{tabular}{rccccccc}
        \tiny\textbf{Metrik} & \tiny\textbf{10 Topics} &\tiny \textbf{40 Topics} &\tiny \textbf{70 Topics} &\tiny \textbf{100 Topics} & \tiny \textbf{130 Topics} &  \tiny \textbf{160 Topics} &  \tiny \textbf{190 Topics} \\
        \hline
        \tiny \textbf{Mean Precision Score}  & \tiny 7.26& \tiny 9.31& \tiny \textbf{10.5}&\tiny 7.31&\tiny 8.51&\tiny 10.4&\tiny 8.72\\
        \tiny \textbf{Mean Recall Score}     & \tiny 0.01& \tiny 0.06& \tiny 0.06&\tiny \textbf{3.89}&\tiny 0.07&\tiny 0.05&\tiny 0.06\\
        \tiny \textbf{Macro F1 Score}        & \tiny 0.02& \tiny 0.12& \tiny 0.12&\tiny \textbf{5.08}&\tiny 0.14&\tiny 0.10&\tiny 0.12\\
        \tiny \textbf{Micro Precision Score} & \tiny 0.01& \tiny 0.08& \tiny 0.08&\tiny \textbf{11.7}&\tiny 0.08&\tiny 0.06&\tiny 0.07\\
        \tiny \textbf{Micro Recall Score}    & \tiny 0.01& \tiny 0.06& \tiny 0.06&\tiny \textbf{9.82}&\tiny 0.06&\tiny 0.05&\tiny 0.06\\
        \tiny \textbf{Micro F1 Score}        & \tiny 0.01& \tiny 0.07& \tiny 0.07&\tiny \textbf{10.6}&\tiny 0.07&\tiny 0.05&\tiny 0.06\\
    \end{tabular}
    \caption{Ergebnisse der SVM mit Topics und One-Vs-Rest}
    \label{tab:topics_svm_ovr}
\end{table}
%Ergebisse darstellen
%Diese werden erst im Kapitel Diskussion interpretiert

\section{Diskussion}

\subsection{Unigram SVM}
% Komentar zu den Unigram SVMs

\subsection{Unigram SVM mit Topicmodells}
% Komentar zu den Topicmodells
Es wird beobachtet, dass die SVM mit Topics und Labelkombinationen deutlich bessere Ergebnisse liefert als die eine mit One-vs-Rest Klassifizierer.
Bei der Klassifizierung mit Labelkombinationen gibt es keine gewaltigen Unterschiede bei den verschiedenen Topicmodells.
Jedoch bemerken wir, dass bei einer nach der Klassengrößen gewichteten Bildung der Durchschnittswerte sich das Modell mit 100 Topics abhebt.
% One-vs-Rest
% vgl zwischen beiden

\subsection{Vergleich zwischen den beiden Verfahren}
% Vergleich zwischen beiden Ansätzen
Wenn wir die Ergebnisse beider Ansätze mit einander vergleichen, lässt sich feststellen, dass die Unigram SVMs um mehr als das fünf-fache besser abschneiden, als die SVM auf dem besten Topicmodell
(je nach Gewichtung, das Modell mit 70 oder mit 100 Topics).


% Probleme
% Speicherplatz (zb mit steigender Anzahl von Topics)

\section{Zusammenfassung}

In dieser Arbeit haben wir zwei Ansätze zur Multilabel-Klassifikation von wissenschaftlichen Publikationen anhand von deren Titeln und Abstracts implementiert und verglichen.
Zunächst wurden die Publikationen direkt mithilfe einer Unigram SVM klassifiziert.
Hier haben wir sowohl eine Labelkombinationen als auch eine One-vs-Rest SVM gelernt und festgestellt, dass die Labelkombinationen SVM, mit einer Micro F1-Score von $61.8 \%$ auf dem Testdaten, bessere Ergebnisse liefert.
Später wurde der Zwischenschritt einer Dimensionsreduktion mit LDA implementiert.
Obwohl es mehrere Modelle mit unterschiedlicher Topicanzahl gelernt wurden, waren die Ergebnisse der Klassifizierung bei diesem Ansatz deutlich schlechter als bei der Unigram SVM.
Die Hypothese, dass eine Dimensionsreduktion der Daten mittels LDA Topicmodells das Klassifikationsproblem optimieren wird, hat sich nicht bestätigt.

% was ist hierbei rausgekommen?
% welches war besser?
% was war unsere vermutung, hat sie sich bestätigt?
% ausblick: wofür ist das gut? kann man das iwie weiter verbessern? oder vielleicht in eine andere richtung gehen?


%\section{Ergebnisse}
%\section{Diskussion}
%\section{Zusammenfassung}

\nocite{*}
\newpage
\bibliography{literature}
\bibliographystyle{alpha}

\end{document}
